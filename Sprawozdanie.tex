
\documentclass[11pt,a4paper]{article}

\usepackage[T1]{fontenc}
\usepackage{graphicx}
\usepackage[polish]{babel}
\usepackage[utf8]{inputenc}
\usepackage{indentfirst}

\usepackage{placeins}
\usepackage{authblk}
\usepackage{siunitx}

\usepackage{filecontents}

%opening

\title{Regresja ceny telewizorów}
\author{Mateusz Jakubczak}

\affil{Wydział Zarządzania, Akademia Górniczo-Hutnicza im. Stanisława Staszica w Krakowie}
\affil{II rok Informatyka i Ekonometria }
\nonstopmode


\begin{document}

\maketitle

\begin{abstract}
	Celem projektu jest zbadanie, jakie parametry telewizora wpływają na jego cenę oraz w jakim stopniu.
	Główną hipotezą badawczą jest stwierdzenie czy cena telewizora jest wprost proporcjonalnie zależna od przekątnej ekranu. 
	Drugą hipotezą jest stwierdzenie czy telewizor używany jest tańszy od nowego.
	Trzecią hipotezą jest stwierdzenie czy jakość ma wpływ na cenę telewizora.
	
\end{abstract}
\newpage

\tableofcontents
\newpage

\section{Opis Danych i ich pochodzenie}

	\subsection{Źródło pochodzenia}
	     Źródłem pochodzenia danych są oferty na stronie allegro.pl dotyczące sprzedaży telewizorów,
		 dane zostały zebrane w dniu \date{31-05-2020}, wszystkie linki znajdują się w pliku "linki.csv".
		 Możliwe, że cześć linków jest już nieaktywna, dlatego wszystkie informacje z tych linków zostały zapisane w pliku 
		 "parametry.csv".
		 
	\subsection{Sposób pobierania}
		Sposób pobierania danych to webscaping informacji z linku za pomocą biblioteki  
		BeautifulSoup w języku programowania Python. Metoda to znalezienie znacznika na cenę telewizora i zapis tej wartości. 
		Potem wyszukuję tablice z parametrami telewizora i dla zbioru moich zmiennych, które wcześniej wybrałem. Jeśli w 
		tabeli parametr zostanie znaleziony, to jego wartość jest zapisywana. W przypadku braku znalezienia określonego parametru zostaje przypisane -1 jako wartość parametru. 
		\begin{figure}[h]
			\centering
			\includegraphics[width=1\linewidth]{figs/strona}
			\caption[Przykładowa tabela z parametrami]{Przykładowa tabela z parametrami}
			\label{fig:strona}
		\end{figure}
	
		Wybrane przez mnie parametry to 
		\begin{itemize}
			\item Stan
			\item Typ telewizora
			\item  Marka
			\item Technologia 3D
			\item  Przekątna ekranu
			\item 	Format HD
			\item 	 Rozdzielczość ekranu
			\item 	Liczba złączy HDMI
			\item 	Technologia HDR
			\item 	Klasa efektywności
			\item Pobór mocy
			\item 	Waga produktu
		\end{itemize}

		
	
\section{Statystyki Opisowe}	
	\subsection{Preprocesing}
		Wszystkie dane zostały wcześniej przygotowane, czyli nadano im odpowiedni typ oraz format, żeby były gotowe do użycia w modelu. Ważniejsza zmiana to przetransformowanie  ,,rozdzielczość ekranu'' oraz ,,format HD'' w zmienną jakość. Każda marka z częstością występowania poniżej mediany została zamieniona na inną, żeby ograniczyć ilość kategorii o małej populacji w zmiennej ,,marka''. Usunięta na starcie została zmienna ,,waga'' ze względu na dużą liczbę brakujących danych. 
		
	\subsection{Zmienna endogeniczna}
		Zmienną endogeniczną jest cena danego telewizora. 
		\begin{figure}[h]
			\centering
			\includegraphics{figs/Cena}
			\caption[Cena]{Wykres pudełkowy dla ceny}
			\label{fig:cena}
		\end{figure}
	
		\begin{center}
			\begin{tabular}{lr}
				 
				{} &          Cena \\
				 
				count &     81.00 \\
				mean  &   2894.43 \\
				std   &   4362.52 \\
				min   &    150.00 \\
				25\%   &    649.00 \\
				50\%   &   1299.00 \\
				75\%   &   3299.00 \\
				max   &  24000.00 \\
				 
			\end{tabular}
		\end{center}
		Mamy wysoką wartość odchyleń standardowych w porównaniu do wartości średniej, mówi to nam o dużej liczbie obserwacji odstających, które należałoby usunąć. 
	
	\subsection{Zmienne ciągłe}
	
		\subsubsection{Przekątna ekranu}
			Zmienna ta opisuje przekątną ekranu telewizora podaną w calach.
			\begin{figure}[h]
				\centering
				\includegraphics{figs/przekatna}
				\caption[Przekątna]{Wykres pudełkowy przekątnej ekranu}
				\label{fig:Przekątna}
			\end{figure}
		
			\begin{center}
				\begin{tabular}{lr}
					 
					{} &  Przekątna ekranu \\
					 
					count &         81.00 \\
					mean  &         48.93 \\
					std   &         15.90 \\
					min   &         20.00 \\
					25\%   &         40.00 \\
					50\%   &         48.000 \\
					75\%   &         60.00 \\
					max   &         82.00 \\
					 
				\end{tabular}
			\end{center}
			
			Nie widać tu obserwacji odstających i wygląda, że wszystko z tą zmienną jest w porządku.
		
		\subsubsection{Pobór mocy}
			Zmienna ta opisuje ile mocy w wat pobiera telewizor podczas normalnej pracy.
			\begin{figure}[h]
				\centering
				\includegraphics{figs/Pobormocy}
				\caption[Pobormocy]{Wykres pudełkowy poboru mocy}
				\label{fig:Pobormocy}
			\end{figure}
			\begin{center}
				\begin{tabular}{lr}
					 
					{} &  Pobór mocy \\
					 
					count &   81.000000 \\
					mean  &  114.518519 \\
					std   &  107.063429 \\
					min   &    1.000000 \\
					25\%   &   42.000000 \\
					50\%   &   90.000000 \\
					75\%   &  141.000000 \\
					max   &  579.000000 \\
					 
				\end{tabular}
			\end{center}
			Mamy tutaj podobną sytuację co w cenie: dużo obserwacji odstających skierowanych w jedną stronę
			oraz wysokie odchylenie standardowe. Sugeruje to usunięcie obserwacji odstających. 
		\subsubsection{złącza HDMI}
			Zmienna ta opisuje ile złączy HDMI posiada dany telewizor
			\begin{figure}[h]
				\centering
				\includegraphics{figs/HDMI}
				\caption[HDMI]{Wykres pudełkowy HDMI}
				\label{fig:HDMI}
			\end{figure}
			\begin{center}
				\begin{tabular}{lr}
					 
					{} &  Liczba złączy HDMI \\
					 
					count &           81.000000 \\
					mean  &            2.975309 \\
					std   &            0.987108 \\
					min   &            0.000000 \\
					25\%   &            2.000000 \\
					50\%   &            3.000000 \\
					75\%   &            4.000000 \\
					max   &            4.000000 \\
					 
				\end{tabular}
			\end{center}
			Zmienna ta nie powinna być obrazowana jako ciągła, ponieważ mamy tylko 5 możliwych wartości, ale możemy zobaczyć, że
			większość telewizorów ma 2, 3 lub 4 złącza HDMI.  
	
	\subsection{Zmienne kategoryczne}
		
		\subsubsection{Typ Telewizora}
			Zmienna ta opisuje, jaki jest typ wyświetlacza w telewizorze.
			\begin{figure}[!htbp]
				\centering
				\includegraphics{figs/Typ}
				\caption[Typ]{Częstość występowania poszczególnych typów}
				\label{fig:typ}
			\end{figure}
			\FloatBarrier
			\newpage
		\subsubsection{Jakość}
			Zmienna ta składa się połączenia dwóch zmiennych(rozdzielczości ekranu oraz formatu obrazu)
			 w procesie preprocesingu.
			\begin{figure}[!htbp]
				\centering
				\includegraphics{figs/jakosc}
				\caption[Jakość]{Częstość występowania poszczególnych jakości}
				\label{fig:jakosc}
			\end{figure}
			\FloatBarrier	
			\newpage
		\subsubsection{Klasa efektywności}
			Zmienna ta opisuje, jakiej klasy efektywności jest telewizor.
			\begin{figure}[!htbp]
				\centering
				\includegraphics{figs/Klasa}
				\caption[Klasa]{Częstość występowania poszczególnych klas}
				\label{fig:Klasa}
			\end{figure}
			\FloatBarrier
			\newpage
		
		\subsubsection{Marka}
			Zmienna ta opisuje, jakiej marki jest telewizor.
			\begin{figure}[!htbp]
				\centering
				\includegraphics{figs/Marka}
				\caption[Marka]{Częstość występowania poszczególnych marek}
				\label{fig:Marka}
			\end{figure}
			\FloatBarrier
			\newpage
		\subsubsection{Technologia 3D}
			Zmienna ta mówi, czy telewizor może wyświetlać obraz w 3D.
			\begin{figure}[!htbp]
				\centering
				\includegraphics{figs/Technologia3D}
				\caption[D3]{Częstość występowania technologii 3D}
				\label{fig:Technologiia3D}
			\end{figure}
			\FloatBarrier
			\newpage
			
		\subsubsection{Technologia HDR}
			Zmienna ta mówi, czy telewizor posiada technologie HDR. Technologia HDR daje lepszą jakość obrazu bez zmiany rozdzielczości. 
			
			\begin{figure}[!htbp]
				\centering
				\includegraphics{figs/HDR}
				\caption[HDR]{Częstość występowania technologii HDR}
				\label{fig:HDR}
			\end{figure}
			\FloatBarrier
			\newpage
			
		
	\subsection{Macierz korelacji}

		\begin{figure}[h]
			\centering
			\includegraphics[width=1.2\linewidth]{figs/macierzkorelacji}
			\caption[macierz]{Macierz korelacji}
			\label{fig:macierzkorelacji}
		\end{figure}
			\FloatBarrier
		\newpage
		Macierz ta przedstawia korelacje Pearsona dla zmiennych ciągłych
 		,,Correlation Ratio'' dla zmiennych kategoryczna z ciągłą
 		Korelacje V Camera dla zmiennych kategorycznych. 
		Z macierzy wynika duża korelacja między ceną, a przekątną, jakością oraz typem telewizora oraz brak korelacji technologii 3D z pozostałymi, co sugeruje niską istotność tego parametru.
		Inne ciekawe korelacje to między typem telewizora, a poborem mocy oraz między jakością, a przekątną, co może sugerować, że jedna z tych zmiennych może nie dostarczyć dotykowych informacji w modelu. Na podstawie tej macierzy możemy przewidzieć, że w modelu powinny znaleźć się zmienne: jakość lub przekątna (jedno z nich z uwagi na wysoką korelację między nimi), typ telewizora oraz marka.
		\newpage
		

\section{Budowa modelu liniowego}

	
	\subsection{Wstępne szacowanie}
		Szacuję pierwszy model z usuniętymi obserwacjami odstającymi, które zostały usunięte na podstawie ceny. 
		Zamienne kategoryczne są reprezentowane jako dummy.
	
		\begin{center}
			\begin{tabular}{lclc}
				 
				\textbf{Dep. Variable:}           &       Cena       & \textbf{  R-squared:         } &     0.907   \\
				\textbf{Model:}                   &       OLS        & \textbf{  Adj. R-squared:    } &     0.856   \\
				\textbf{Method:}                  &  Least Squares   & \textbf{  F-statistic:       } &     17.91   \\
				\textbf{Date:}                    & Sat, 13 Jun 2020 & \textbf{  Prob (F-statistic):} &  1.37e-15   \\
				\textbf{Time:}                    &     18:13:48     & \textbf{  Log-Likelihood:    } &   -498.25   \\
				\textbf{No. Observations:}        &          69      & \textbf{  AIC:               } &     1047.   \\
				\textbf{Df Residuals:}            &          44      & \textbf{  BIC:               } &     1102.   \\
				\textbf{Df Model:}                &          24      & \textbf{                     } &             \\
				 
			\end{tabular}
			\begin{tabular}{lcccccc}
				& \textbf{coef} & \textbf{std err} & \textbf{t} & \textbf{P$> |$t$|$} & \textbf{[0.025} & \textbf{0.975]}  \\
				 
				\textbf{const}                    &    -793.2401  &      502.552     &    -1.578  &         0.122        &    -1806.068    &      219.588     \\
				\textbf{Przekątna ekranu}         &      46.2102  &        8.122     &     5.690  &         0.000        &       29.842    &       62.579     \\
				\textbf{Liczba złączy HDMI}       &      -2.8332  &       88.866     &    -0.032  &         0.975        &     -181.932    &      176.265     \\
				\textbf{Pobór mocy}               &       0.4069  &        1.287     &     0.316  &         0.753        &       -2.187    &        3.001     \\
				\textbf{Stan\_Po zwrocie}         &    -379.1746  &      348.611     &    -1.088  &         0.283        &    -1081.755    &      323.406     \\
				\textbf{Stan\_Powystawowy}        &     391.6052  &      327.285     &     1.197  &         0.238        &     -267.995    &     1051.205     \\
				\textbf{Stan\_U¿ywany}            &    -337.6844  &      190.170     &    -1.776  &         0.083        &     -720.946    &       45.577     \\
				\textbf{Typ telewizora\_LED}      &     -41.0985  &      246.994     &    -0.166  &         0.869        &     -538.882    &      456.686     \\
				\textbf{Typ telewizora\_OLED}     &    -256.0298  &      581.007     &    -0.441  &         0.662        &    -1426.973    &      914.913     \\
				\textbf{Typ telewizora\_QLED}     &     908.7689  &      365.592     &     2.486  &         0.017        &      171.966    &     1645.572     \\
				\textbf{Typ telewizora\_inny}     &     391.9251  &      509.150     &     0.770  &         0.446        &     -634.199    &     1418.049     \\
				\textbf{Typ telewizora\_plazmowy} &    -457.9512  &      437.322     &    -1.047  &         0.301        &    -1339.317    &      423.414     \\
				\textbf{Marka\_LG}                &     442.5958  &      282.264     &     1.568  &         0.124        &     -126.271    &     1011.462     \\
				\textbf{Marka\_Philips}           &      99.4449  &      356.995     &     0.279  &         0.782        &     -620.030    &      818.920     \\
				\textbf{Marka\_Samsung}           &     333.8736  &      296.659     &     1.125  &         0.267        &     -264.002    &      931.750     \\
				\textbf{Marka\_Sony}              &     292.2040  &      321.864     &     0.908  &         0.369        &     -356.471    &      940.879     \\
				\textbf{Marka\_inna}              &      86.5933  &      228.325     &     0.379  &         0.706        &     -373.566    &      546.753     \\
				\textbf{Technologiia 3D\_True}     &     123.1922  &      296.478     &     0.416  &         0.680        &     -474.320    &      720.705     \\
				\textbf{Technologiia HDR\_Tak}     &      62.7285  &      252.602     &     0.248  &         0.805        &     -446.357    &      571.814     \\
				\textbf{x \_A+}   &      27.3171  &      134.129     &     0.204  &         0.840        &     -243.002    &      297.636     \\
				\textbf{Klasa efektywnosci\_B}    &    -263.0567  &      491.918     &    -0.535  &         0.596        &    -1254.451    &      728.338     \\
				\textbf{Klasa efektywnosci\_C}    &     782.4736  &      587.331     &     1.332  &         0.190        &     -401.214    &     1966.161     \\
				\textbf{Klasa efektywnosci\_D}    &   -1.208e-12  &     6.09e-12     &    -0.198  &         0.844        &    -1.35e-11    &     1.11e-11     \\
				\textbf{Klasa efektywnosci\_inna} &    -190.0021  &      210.011     &    -0.905  &         0.371        &     -613.251    &      233.246     \\
				\textbf{jakosc\_Full HD}          &    -120.3698  &      227.939     &    -0.528  &         0.600        &     -579.750    &      339.010     \\
				\textbf{jakosc\_HD Ready}         &    -110.9307  &      263.460     &    -0.421  &         0.676        &     -641.900    &      420.038     \\
				 
			\end{tabular}
			\begin{tabular}{lclc}
				\textbf{Omnibus:}       &  2.845 & \textbf{  Durbin-Watson:     } &    2.125  \\
				\textbf{Prob(Omnibus):} &  0.241 & \textbf{  Jarque-Bera (JB):  } &    2.030  \\
				\textbf{Skew:}          &  0.329 & \textbf{  Prob(JB):          } &    0.362  \\
				\textbf{Kurtosis:}      &  3.523 & \textbf{  Cond. No.          } & 1.23e+16  \\
				 
			\end{tabular}
			%\caption{OLS Regression Results}
		\end{center}
		

		 \begin{figure}[h]
		 	\centering
		 	\includegraphics[width=1\linewidth]{figs/wtepna_analiza}
		 	\caption[wstepna]{Wykres dopasowania modelu do ceny na postawie przekątnej ekranu}
		 	\label{fig:wtepnaanaliza}
		 \end{figure}
	Wysokie wartości p-value dla zmiennych wskazują na ich niepotrzebność w modelu. Na podstawie wartości statystki 
	 Durbin-Watson nie występuje autokorelacja. P-value Jarque-Bera wskazuje, że nie mamy spełnionego założenia o normalności reszt. Obecna postać modelu jest niepoprawna i musimy ją odrzucić. 
	 
		 \begin{center}
			 \begin{tabular}{llr}
			 	 
		 	{} &                  	Zmienna &           VIF \\
			 	 
			 	0  &                    const &  101.425957 \\
			 	1  &         Przekątna ekranu &    4.551214 \\
			 	2  &       Liczba złączy HDMI &    2.966977 \\
			 	3  &               Pobór mocy &    3.756607 \\
			 	4  &          Stan\_Po zwrocie &    1.373651 \\
			 	5  &         Stan\_Powystawowy &    1.788984 \\
			 	6  &             Stan\_Używany &    3.623989 \\
			 	7  &       Typ telewizora\_LED &    5.042987 \\
			 	8  &      Typ telewizora\_OLED &    1.936244 \\
			 	9  &      Typ telewizora\_QLED &    6.088026 \\
			 	10 &      Typ telewizora\_inny &    1.486924 \\
			 	11 &  Typ telewizora\_plazmowy &    4.194356 \\
			 	12 &                 Marka\_LG &    4.596809 \\
			 	13 &            Marka\_Philips &    3.440026 \\
			 	14 &            Marka\_Samsung &    7.274918 \\
			 	15 &               Marka\_Sony &    3.303129 \\
			 	16 &               Marka\_inna &    4.309457 \\
			 	17 &      Technologiia 3D\_True &    2.372596 \\
			 	18 &      Technologiia HDR\_Tak &    5.812804 \\
			 	19 &    Klasa efektywnosci\_A+ &    1.441642 \\
			 	20 &     Klasa efektywnosci\_B &    1.387976 \\
			 	21 &     Klasa efektywnosci\_C &    1.978623 \\
			 	22 &     Klasa efektywnosci\_D &         NaN \\
			 	23 &  Klasa efektywnosci\_inna &    2.194941 \\
			 	24 &           jakosc\_Full HD &    4.531532 \\
			 	25 &          jakosc\_HD Ready &    5.374781 \\
			 	 
			 \end{tabular}
		 \end{center}
	 	Na podstawie statystyki VIF nie mamy rażącej współliniowości między zmiennymi, wiec nie ma podstaw do odrzucenia zmiennych na podstawie wysokiej współliniowości.

 
	
	\subsection{Iteratywne poprawnie modelu}
		Do wyboru zmiennych używam metody krokowej wstecznej, przyjęta wartość $\alpha$ to $0,1$. Kolejne zmienne usunięte ze zbioru danych to:\\
		Liczba złączy HDMI, z p-value równe 0.97\\
		jakość, z p-value równe 0.86\\
		Technologiia 3D, z p-value równe 0.79\\
		Pobór mocy, z p-value równe 0.72\\
		Technologiia HDR, z p-value równe 0.51\\
		Marka, z p-value równe 0.20\\
		Klasa efektywności, z p-value równe 0.16\\
		Wartości p-value to p-value F statystki wszystkich dummy zmiennych z kategorii lub wartość t-testu dla zmiennej. \\
		
		Otrzymany w ten sposób model wygląda tak:\\
		\begin{center}
			\begin{tabular}{lclc}
				 
				\textbf{Dep. Variable:}           &       Cena       & \textbf{  R-squared:         } &     0.878   \\
				\textbf{Model:}                   &       OLS        & \textbf{  Adj. R-squared:    } &     0.859   \\
				\textbf{Method:}                  &  Least Squares   & \textbf{  F-statistic:       } &     47.02   \\
				\textbf{Date:}                    & Sat, 13 Jun 2020 & \textbf{  Prob (F-statistic):} &  1.21e-23   \\
				\textbf{Time:}                    &     18:13:49     & \textbf{  Log-Likelihood:    } &   -507.77   \\
				\textbf{No. Observations:}        &          69      & \textbf{  AIC:               } &     1036.   \\
				\textbf{Df Residuals:}            &          59      & \textbf{  BIC:               } &     1058.   \\
				\textbf{Df Model:}                &           9      & \textbf{                     } &             \\
				 
			\end{tabular}
			\begin{tabular}{lcccccc}
				& \textbf{coef} & \textbf{std err} & \textbf{t} & \textbf{P$> |$t$|$} & \textbf{[0.025} & \textbf{0.975]}  \\
				 
				\textbf{const}                    &   -1150.0660  &      262.728     &    -4.377  &         0.000        &    -1675.784    &     -624.348     \\
				\textbf{Przekątna ekranu}         &      56.5314  &        4.794     &    11.793  &         0.000        &       46.939    &       66.123     \\
				\textbf{Stan\_Po zwrocie}         &    -265.3156  &      306.228     &    -0.866  &         0.390        &     -878.077    &      347.446     \\
				\textbf{Stan\_Powystawowy}        &     480.0764  &      311.772     &     1.540  &         0.129        &     -143.778    &     1103.931     \\
				\textbf{Stan\_Używany}            &    -305.4451  &      116.852     &    -2.614  &         0.011        &     -539.266    &      -71.624     \\
				\textbf{Typ telewizora\_LED}      &      21.4830  &      209.644     &     0.102  &         0.919        &     -398.014    &      440.980     \\
				\textbf{Typ telewizora\_OLED}     &    -105.5515  &      554.635     &    -0.190  &         0.850        &    -1215.374    &     1004.271     \\
				\textbf{Typ telewizora\_QLED}     &    1058.4140  &      282.623     &     3.745  &         0.000        &      492.887    &     1623.941     \\
				\textbf{Typ telewizora\_inny}     &     461.8203  &      452.754     &     1.020  &         0.312        &     -444.138    &     1367.779     \\
				\textbf{Typ telewizora\_plazmowy} &    -215.3403  &      289.991     &    -0.743  &         0.461        &     -795.611    &      364.931     \\
				 
			\end{tabular}
			\begin{tabular}{lclc}
				\textbf{Omnibus:}       &  5.021 & \textbf{  Durbin-Watson:     } &    2.038  \\
				\textbf{Prob(Omnibus):} &  0.081 & \textbf{  Jarque-Bera (JB):  } &    6.855  \\
				\textbf{Skew:}          & -0.016 & \textbf{  Prob(JB):          } &   0.0325  \\
				\textbf{Kurtosis:}      &  4.544 & \textbf{  Cond. No.          } &     591.  \\
				 
			\end{tabular}
			%\caption{OLS Regression Results}
		\end{center}
		
		Model ten spełnia podstawowe założenia. W modelu nie ma katalizatorów.  \\
		Model nie spełnia testu serii  z wynikiem p-value = 0.82, co z dużą pewnością odrzuca $H_0$ o liniowości modelu.\\
		
	\textbf{	Oszacujmy nowy model dla logarytmu ceny }
		
		\begin{center}
			\begin{tabular}{lclc}
				 
				\textbf{Dep. Variable:}           &       Cena       & \textbf{  R-squared:         } &     0.825   \\
				\textbf{Model:}                   &       OLS        & \textbf{  Adj. R-squared:    } &     0.798   \\
				\textbf{Method:}                  &  Least Squares   & \textbf{  F-statistic:       } &     30.81   \\
				\textbf{Date:}                    & Sat, 13 Jun 2020 & \textbf{  Prob (F-statistic):} &  4.06e-19   \\
				\textbf{Time:}                    &     18:13:49     & \textbf{  Log-Likelihood:    } &   -22.273   \\
				\textbf{No. Observations:}        &          69      & \textbf{  AIC:               } &     64.55   \\
				\textbf{Df Residuals:}            &          59      & \textbf{  BIC:               } &     86.89   \\
				\textbf{Df Model:}                &           9      & \textbf{                     } &             \\
				 
			\end{tabular}
			\begin{tabular}{lcccccc}
				& \textbf{coef} & \textbf{std err} & \textbf{t} & \textbf{P$> |$t$|$} & \textbf{[0.025} & \textbf{0.975]}  \\
				 
				\textbf{const}                    &       4.5693  &        0.231     &    19.775  &         0.000        &        4.107    &        5.032     \\
				\textbf{Przekątna ekranu}         &       0.0427  &        0.004     &    10.132  &         0.000        &        0.034    &        0.051     \\
				\textbf{Stan\_Po zwrocie}         &       0.0405  &        0.269     &     0.150  &         0.881        &       -0.498    &        0.579     \\
				\textbf{Stan\_Powystawowy}        &       0.0863  &        0.274     &     0.315  &         0.754        &       -0.462    &        0.635     \\
				\textbf{Stan\_Używany}            &      -0.1141  &        0.103     &    -1.110  &         0.272        &       -0.320    &        0.092     \\
				\textbf{Typ telewizora\_LED}      &       0.4988  &        0.184     &     2.705  &         0.009        &        0.130    &        0.868     \\
				\textbf{Typ telewizora\_OLED}     &       0.5401  &        0.488     &     1.107  &         0.273        &       -0.436    &        1.516     \\
				\textbf{Typ telewizora\_QLED}     &       0.9832  &        0.249     &     3.956  &         0.000        &        0.486    &        1.481     \\
				\textbf{Typ telewizora\_inny}     &       0.1265  &        0.398     &     0.318  &         0.752        &       -0.670    &        0.923     \\
				\textbf{Typ telewizora\_plazmowy} &       0.2606  &        0.255     &     1.022  &         0.311        &       -0.250    &        0.771     \\
				 
			\end{tabular}
			\begin{tabular}{lclc}
				\textbf{Omnibus:}       & 31.663 & \textbf{  Durbin-Watson:     } &    1.856  \\
				\textbf{Prob(Omnibus):} &  0.000 & \textbf{  Jarque-Bera (JB):  } &   86.469  \\
				\textbf{Skew:}          & -1.394 & \textbf{  Prob(JB):          } & 1.67e-19  \\
				\textbf{Kurtosis:}      &  7.723 & \textbf{  Cond. No.          } &     591.  \\
				 
			\end{tabular}
			%\caption{OLS Regression Results}
		\end{center}
	
		Model ten spełnia poprzednie założenia. \\
		
		Otrzymujemy wysokie p-value dla testu RESET, p-value = 0.67, prowadzi to do odrzucenia $H_0$ o poprawności obecnego modelu.
		
		Test ten przyjmuje tylko zmienne ciągłe a jedyną zmienną ciągłą w modelu, jest przekątna ekranu. 
		Więc dokonamy transformacji tej zmiennej, podnosząc ją do kwadratu.
		
		
			\textbf{Oszacujmy nowy model z kwadratem zmiennej przekątna ekranu }
			
			\begin{center}
				\begin{tabular}{lclc}
					 
					\textbf{Dep. Variable:}           &       Cena       & \textbf{  R-squared:         } &     0.805   \\
					\textbf{Model:}                   &       OLS        & \textbf{  Adj. R-squared:    } &     0.775   \\
					\textbf{Method:}                  &  Least Squares   & \textbf{  F-statistic:       } &     27.08   \\
					\textbf{Date:}                    & Sat, 13 Jun 2020 & \textbf{  Prob (F-statistic):} &  8.33e-18   \\
					\textbf{Time:}                    &     18:13:49     & \textbf{  Log-Likelihood:    } &   -25.902   \\
					\textbf{No. Observations:}        &          69      & \textbf{  AIC:               } &     71.80   \\
					\textbf{Df Residuals:}            &          59      & \textbf{  BIC:               } &     94.14   \\
					\textbf{Df Model:}                &           9      & \textbf{                     } &             \\
					 
				\end{tabular}
				\begin{tabular}{lcccccc}
					& \textbf{coef} & \textbf{std err} & \textbf{t} & \textbf{P$> |$t$|$} & \textbf{[0.025} & \textbf{0.975]}  \\
					 
					\textbf{const}                    &       5.3786  &        0.209     &    25.754  &         0.000        &        4.961    &        5.797     \\
					\textbf{Przek¹tna ekranu}         &       0.0004  &      4.7e-05     &     9.302  &         0.000        &        0.000    &        0.001     \\
					\textbf{Stan\_Po zwrocie}         &       0.1391  &        0.282     &     0.492  &         0.624        &       -0.426    &        0.704     \\
					\textbf{Stan\_Powystawowy}        &       0.0632  &        0.290     &     0.218  &         0.828        &       -0.517    &        0.644     \\
					\textbf{Stan\_U¿ywany}            &      -0.0465  &        0.108     &    -0.430  &         0.669        &       -0.263    &        0.170     \\
					\textbf{Typ telewizora\_LED}      &       0.6247  &        0.192     &     3.259  &         0.002        &        0.241    &        1.008     \\
					\textbf{Typ telewizora\_OLED}     &       0.6817  &        0.512     &     1.331  &         0.188        &       -0.343    &        1.707     \\
					\textbf{Typ telewizora\_QLED}     &       1.1406  &        0.256     &     4.449  &         0.000        &        0.628    &        1.654     \\
					\textbf{Typ telewizora\_inny}     &      -0.0224  &        0.418     &    -0.054  &         0.957        &       -0.860    &        0.815     \\
					\textbf{Typ telewizora\_plazmowy} &       0.3887  &        0.266     &     1.464  &         0.149        &       -0.143    &        0.920     \\
					 
				\end{tabular}
				\begin{tabular}{lclc}
					\textbf{Omnibus:}       & 28.917 & \textbf{  Durbin-Watson:     } &    1.829  \\
					\textbf{Prob(Omnibus):} &  0.000 & \textbf{  Jarque-Bera (JB):  } &   70.967  \\
					\textbf{Skew:}          & -1.308 & \textbf{  Prob(JB):          } & 3.89e-16  \\
					\textbf{Kurtosis:}      &  7.224 & \textbf{  Cond. No.          } & 3.17e+04  \\
					 
				\end{tabular}
				%\caption{OLS Regression Results}
			\end{center}
		
			Model ten spełnia poprzednie założenia, w tym nie ma podstaw do odrzucenia $H_0$ w teście RESET, co oznacza poprawną postać modelu. \\
			
			Zarówno test Breusch–Pagana jak i test Whita daje podstawy do odrzucenia $H_0$ o homdeskatyczności modelu.\\
			
			Wykonujemy więc standardową korektę na heteroskedastyczność i otrzymujemy model w postaci: 
			\begin{center}
				\begin{tabular}{lclc}
					 
					\textbf{Dep. Variable:}           &        y         & \textbf{  R-squared:         } &     0.809   \\
					\textbf{Model:}                   &       OLS        & \textbf{  Adj. R-squared:    } &     0.780   \\
					\textbf{Method:}                  &  Least Squares   & \textbf{  F-statistic:       } &     27.76   \\
					\textbf{Date:}                    & Sat, 13 Jun 2020 & \textbf{  Prob (F-statistic):} &  4.69e-18   \\
					\textbf{Time:}                    &     18:13:49     & \textbf{  Log-Likelihood:    } &   -17.733   \\
					\textbf{No. Observations:}        &          69      & \textbf{  AIC:               } &     55.47   \\
					\textbf{Df Residuals:}            &          59      & \textbf{  BIC:               } &     77.81   \\
					\textbf{Df Model:}                &           9      & \textbf{                     } &             \\
					 
				\end{tabular}
				\begin{tabular}{lcccccc}
					& \textbf{coef} & \textbf{std err} & \textbf{t} & \textbf{P$> |$t$|$} & \textbf{[0.025} & \textbf{0.975]}  \\
					 
					\textbf{const}                    &       5.6993  &        0.181     &    31.483  &         0.000        &        5.337    &        6.061     \\
					\textbf{Przek¹tna ekranu}         &       0.0002  &     2.36e-05     &     7.116  &         0.000        &        0.000    &        0.000     \\
					\textbf{Stan\_Po zwrocie}         &       0.3620  &        0.249     &     1.452  &         0.152        &       -0.137    &        0.861     \\
					\textbf{Stan\_Powystawowy}        &       0.4769  &        0.252     &     1.891  &         0.064        &       -0.028    &        0.982     \\
					\textbf{Stan\_U¿ywany}            &      -0.0655  &        0.096     &    -0.681  &         0.498        &       -0.258    &        0.127     \\
					\textbf{Typ telewizora\_LED}      &       0.8279  &        0.168     &     4.921  &         0.000        &        0.491    &        1.165     \\
					\textbf{Typ telewizora\_OLED}     &       1.0879  &        0.452     &     2.406  &         0.019        &        0.183    &        1.992     \\
					\textbf{Typ telewizora\_QLED}     &       1.6965  &        0.216     &     7.853  &         0.000        &        1.264    &        2.129     \\
					\textbf{Typ telewizora\_inny}     &      -0.1934  &        0.371     &    -0.521  &         0.604        &       -0.936    &        0.549     \\
					\textbf{Typ telewizora\_plazmowy} &       0.5140  &        0.240     &     2.144  &         0.036        &        0.034    &        0.994     \\
					 
				\end{tabular}
				\begin{tabular}{lclc}
					\textbf{Omnibus:}       &  6.372 & \textbf{  Durbin-Watson:     } &    1.969  \\
					\textbf{Prob(Omnibus):} &  0.041 & \textbf{  Jarque-Bera (JB):  } &   10.243  \\
					\textbf{Skew:}          & -0.117 & \textbf{  Prob(JB):          } &  0.00597  \\
					\textbf{Kurtosis:}      &  4.873 & \textbf{  Cond. No.          } & 3.86e+04  \\
					 
				\end{tabular}
				%\caption{OLS Regression Results}
			\end{center}
		
			Model ten spełnia poprzednie założenia i jest homdeskatyczny.
			
			Przeprowadzając test Chowa, dzieląc losowa dane na połowę, zauważamy, że model, jest wrażliwy na dobór danych.
			\begin{center}
				\begin{tabular}{lr}
						 
				Numer próby &     P-value dla testu Chowa \\
						 
						0  &  \num{2.045944e-01} \\
						1  & \num{ 6.442818e-04} \\
						2  & \num{ 4.177916e-02} \\
						3  &  \num{2.064195e-03} \\
						4  &  \num{7.591984e-03} \\
						5  &  \num{3.833329e-01} \\
						6  & \num{ 3.330146e-08} \\
						7  &  \num{1.805170e-01} \\
						8  &  \num{5.618481e-03} \\
						9  &  \num{9.585642e-01} \\
						10 &  \num{4.304746e-01} \\
						11 &  \num{6.677985e-03 }\\
						12 &  \num{1.630223e-01} \\
						13 &  \num{5.255308e-02} \\
						14 &  \num{7.709834e-01} \\
						15 &  \num{2.007603e-01} \\
						16 &  \num{1.769388e-01} \\
						17 &  \num{4.549207e-01} \\
						18 &  \num{2.890782e-01} \\
						19 & \num{ 2.451896e-02} \\
						 
					\end{tabular}
			\end{center}
		Widać tu że mamy wyniki mówiące zarówno o stabilności modelu, jak i jego braku. Wynika to z dużej ilości obserwacji rzadkich , które mają wysoki wpłych na wynik tego testu. Usuwanie danych, żeby doprowadzić do stabilnośći modelu, jest nieuzasadnione i zmniejszy tylko możliwości generalizacji modelu. 
			
	\subsection{Wybór postaci modelu}	
		Ostateczna postać modelu to $$log(\text{Cena}) \sim \alpha_0 + \alpha_1*
										 \text{Przekątna ekranu}^2 + \alpha_2*\text{Stan}_{dummy}
										 +\alpha_2\text{Typ telewizora}_{dummy}$$
		
		
\section{Testowanie własności modelu}

	
	\subsection{Testy Modelu}
	\textbf{Ostateczna wersja modelu to:}
	
	\begin{center}
		\begin{tabular}{lclc}
			 
			\textbf{Dep. Variable:}           &        y         & \textbf{  R-squared:         } &     0.809   \\
			\textbf{Model:}                   &       OLS        & \textbf{  Adj. R-squared:    } &     0.780   \\
			\textbf{Method:}                  &  Least Squares   & \textbf{  F-statistic:       } &     27.76   \\
			\textbf{Date:}                    & Sat, 13 Jun 2020 & \textbf{  Prob (F-statistic):} &  4.69e-18   \\
			\textbf{Time:}                    &     18:13:50     & \textbf{  Log-Likelihood:    } &   -17.733   \\
			\textbf{No. Observations:}        &          69      & \textbf{  AIC:               } &     55.47   \\
			\textbf{Df Residuals:}            &          59      & \textbf{  BIC:               } &     77.81   \\
			\textbf{Df Model:}                &           9      & \textbf{                     } &             \\
			 
		\end{tabular}
		\begin{tabular}{lcccccc}
			& \textbf{coef} & \textbf{std err} & \textbf{t} & \textbf{P$> |$t$|$} & \textbf{[0.025} & \textbf{0.975]}  \\
			 
			\textbf{const}                    &       5.6993  &        0.181     &    31.483  &         0.000        &        5.337    &        6.061     \\
			\textbf{Przekątna ekranu}         &       0.0002  &     2.36e-05     &     7.116  &         0.000        &        0.000    &        0.000     \\
			\textbf{Stan\_Po zwrocie}         &       0.3620  &        0.249     &     1.452  &         0.152        &       -0.137    &        0.861     \\
			\textbf{Stan\_Powystawowy}        &       0.4769  &        0.252     &     1.891  &         0.064        &       -0.028    &        0.982     \\
			\textbf{Stan\_Używany}            &      -0.0655  &        0.096     &    -0.681  &         0.498        &       -0.258    &        0.127     \\
			\textbf{Typ telewizora\_LED}      &       0.8279  &        0.168     &     4.921  &         0.000        &        0.491    &        1.165     \\
			\textbf{Typ telewizora\_OLED}     &       1.0879  &        0.452     &     2.406  &         0.019        &        0.183    &        1.992     \\
			\textbf{Typ telewizora\_QLED}     &       1.6965  &        0.216     &     7.853  &         0.000        &        1.264    &        2.129     \\
			\textbf{Typ telewizora\_inny}     &      -0.1934  &        0.371     &    -0.521  &         0.604        &       -0.936    &        0.549     \\
			\textbf{Typ telewizora\_plazmowy} &       0.5140  &        0.240     &     2.144  &         0.036        &        0.034    &        0.994     \\
			 
	\end{tabular}
		\begin{tabular}{lclc}
			\textbf{Omnibus:}       &  6.372 & \textbf{  Durbin-Watson:     } &    1.969  \\
			\textbf{Prob(Omnibus):} &  0.041 & \textbf{  Jarque-Bera (JB):  } &   10.243  \\
			\textbf{Skew:}          & -0.117 & \textbf{  Prob(JB):          } &  0.00597  \\
			\textbf{Kurtosis:}      &  4.873 & \textbf{  Cond. No.          } & 3.86e+04  \\
			 
		\end{tabular}
		%\caption{OLS Regression Results}
	\end{center}

\resizebox{\columnwidth}{!}{%
	\begin{tabular}{cccc}
		Statystyka/Test & P-value & Komentarz & Decyzja \\ 
		Jarque-Bera  & 0.005 & Rozkład reszt jes normalny & Brak podstaw do odrzucenia $H_0$ \\ 
		F-statistic & \num{4.69e-18} & $R^2$ jest statystycznie znaczący & Brak podstaw do odrzucenia $H_0$ \\ 
		Durbin-Watson & 1.969 & Nie ma autokorelacji & Brak podstaw do odrzucenia $H_0$ \\ 
		Istotnośći zmiennych  & ------ & ------ & ----- \\ 
		F Stan  & 0.04 & Jest znacząca & Brak podstaw do odrzucenia $H_0$ \\ 
		F Typ telewizora & \num{5.88649788e-08} & Jest znacząca & Brak podstaw do odrzucenia $H_0$ \\ 
		t-testPrzekątna ekranu & \num{1.723455164413816e-09} & Jest znacząca & Brak podstaw do odrzucenia $H_0$ \\ 
		t-test const & \num{1.3798049965915669e-38} & Jest znacząca & Brak podstaw do odrzucenia $H_0$ \\ 
		Kataliza & ----- & Brak par katalizatorów & ------ \\ 
		Test serii & 0.00 & Model jest liniowy & Brak podstaw do odrzucenia $H_0$ \\ 
		RESET & 0.0046 & Postać modelu jest poprawna & Brak podstaw do odrzucenia $H_0$ \\ 
		Breusch-Pagan & \num{3.140388461885039e-07} & Model jest homodeksatyczny & Brak podstaw do odrzucenia $H_0$ \\ 
		White & \num{2.2945588282568748e-05} & Model jest homodeksatyczny & Brak podstaw do odrzucenia $H_0$ \\ 
		Chow & -------- & Model nie jest stabliny & ---------- \\ 
		Współiność  & -------- & VIF i test wspóliniowsci wskazuje na jej brak & -------
	\end{tabular} 
}
	\subsection{Interpretacja parametrów modelu.}
		Interpretacja dla Stanu i Typu jest taka sama, z uwagą żeby otrzymać o ile zmienni to wartość w cenie należy użyć ich jako argument w funkcji $e^x$. W Stanie porównuje się do sytuacji, kiedy Stan to nowy, a w typie kiedy typ to LCD.
		
		Interpretacja Przekątnej wraz ze wzrostem przekątnej następuje wzrost ceny. Dokładna wartość nie możliwa do podania ponieważ zależność jest nie liniowa, wzrost ceny o jeden cal z 20 do 21 nie jest równe zmiany o jeden cal z 70 do 71.
	
		
		
	
	\subsection{predykcja wraz z 95\% przedziałem ufności.}
	\begin{tabular}{lrrrr}
	
		{} &  średnia wartosc &        Dolna &         Górna &  Prawdziwa \\
		 
		0  &      2477.346083 &  1174.160943 &   4944.613420 &     3299.0 \\
		1  &      2477.346083 &  1174.160943 &   4944.613420 &     3299.0 \\
		2  &       615.199526 &   378.295259 &   1513.229797 &      399.0 \\
		3  &      1755.097975 &   980.177779 &   4175.716966 &     2178.0 \\
		4  &       720.531276 &   410.408866 &   1636.248821 &      599.0 \\
		5  &       722.793677 &   413.031360 &   1646.376215 &      579.0 \\
		6  &      2899.000000 &  1112.795004 &   7552.335309 &     2899.0 \\
		7  &       749.178958 &   444.101065 &   1767.320037 &      419.0 \\
		8  &      3066.151654 &   963.761964 &   4076.936438 &     4199.0 \\
		9  &      1066.199624 &   471.656930 &   1876.099749 &     1585.0 \\
		10 &      3999.332996 &  1809.964927 &   7623.517395 &     3699.0 \\
		11 &      3999.332996 &  1809.964927 &   7623.517395 &     3699.0 \\
		12 &       744.763607 &   438.840188 &   1746.715312 &      439.0 \\
		13 &       272.200996 &   143.085099 &    631.713298 &      229.0 \\
		14 &       272.200996 &   143.085099 &    631.713298 &      229.0 \\
		15 &       814.179933 &   426.512161 &   1701.673946 &      730.0 \\
		16 &       845.200028 &   439.281724 &   1748.442639 &      879.0 \\
		17 &      1448.601606 &   659.347273 &   3462.556155 &     1349.0 \\
		18 &       720.531276 &   410.408866 &   1636.248821 &      599.0 \\
		19 &       699.036214 &   385.826670 &   1541.917682 &      899.0 \\
		20 &      2273.167078 &   815.195432 &   4368.230180 &     3599.0 \\
		21 &       738.887435 &   431.876923 &   1719.522486 &      469.0 \\
		22 &       715.510799 &   404.613073 &   1613.911042 &      649.0 \\
		23 &      1675.306026 &   700.884229 &   3680.687097 &     1799.0 \\
		24 &      5525.555125 &  2875.423108 &  15407.973975 &     3490.0 \\
		25 &      1104.362366 &   505.899087 &   2013.309502 &     1299.0 \\
		26 &      1091.946134 &   494.657771 &   1968.013483 &     1379.0 \\
		27 &       733.756165 &   425.832241 &   1695.989835 &      499.0 \\
		28 &      2531.585577 &  1831.725541 &  10316.271365 &      399.0 \\
		29 &      3350.748625 &  1132.729444 &   4947.324527 &     3599.0 \\
		30 &      1076.015115 &   480.376057 &   1910.821768 &     1499.0 \\
		31 &       868.620061 &   415.809568 &   1657.118554 &     1449.0 \\
		32 &      4017.392554 &  1825.944816 &   7694.222180 &     3659.0 \\
		33 &       720.531276 &   410.408866 &   1636.248821 &      599.0 \\
		34 &      1272.514533 &   533.530526 &   2125.705859 &     1738.0 \\
		35 &      1238.973706 &   506.064055 &   2013.976073 &     1976.0 \\
		36 &       617.933146 &   381.748610 &   1526.371734 &      367.0 \\
		
		
		\end{tabular}
	
	
		\begin{tabular}{lrrrr}
			 
			{} &  średnia wartosc &        Dolna &         Górna &  Prawdziwa \\
		
		
		37 &       845.439714 &   439.533038 &   1749.425963 &      877.0 \\
		38 &       895.927552 &   442.672925 &   1761.721464 &     1099.0 \\
		39 &      4017.392554 &  1825.944816 &   7694.222180 &     3659.0 \\
		40 &       819.356651 &   431.982467 &   1723.264747 &      699.0 \\
		41 &       823.364341 &   390.792879 &   1562.139778 &     1349.0 \\
		42 &       615.191045 &   363.972238 &   1458.972751 &      454.0 \\
		43 &      1279.535999 &   563.134829 &   2259.325292 &     1299.0 \\
		44 &      1170.000553 &   472.174385 &   1883.748679 &     1899.0 \\
		45 &      1018.861246 &   532.145719 &   2129.410888 &      680.0 \\
		46 &      2464.866227 &  1197.858648 &   5212.556440 &     2499.0 \\
		47 &       827.861431 &   410.303421 &   1638.047400 &      999.0 \\
		48 &       879.389087 &   635.765770 &   2571.885062 &      150.0 \\
		49 &      1030.889548 &   544.520344 &   2181.043383 &      650.0 \\
		50 &      1310.123036 &   589.584036 &   2371.830952 &     1199.0 \\
		51 &      1108.262847 &   532.253140 &   2129.857701 &      950.0 \\
		52 &       414.534153 &   182.718383 &    806.587467 &      418.0 \\
		53 &       866.789114 &   433.550867 &   1729.466087 &      890.0 \\
		54 &       775.967442 &   386.985278 &   1547.411235 &     1100.0 \\
		55 &       769.914551 &   380.864490 &   1523.792456 &     1199.0 \\
		56 &       427.712536 &   194.324963 &    859.551237 &      349.0 \\
		57 &       992.075369 &   310.675252 &   1434.447694 &     2500.0 \\
		58 &       250.000000 &    95.963695 &    651.287971 &      250.0 \\
		59 &       612.680606 &   262.527908 &   1231.152460 &     1000.0 \\
		60 &       826.813141 &   439.909407 &   1754.657774 &      659.0 \\
		61 &       671.617433 &   372.729999 &   1492.512250 &      679.0 \\
		62 &       348.825142 &   153.886683 &    678.476830 &      599.0 \\
		63 &       869.866399 &   486.755660 &   1942.787954 &      499.0 \\
		64 &       706.851938 &   354.604947 &   1423.254348 &     1549.0 \\
		65 &       615.836034 &   265.391430 &   1243.152119 &      950.0 \\
		66 &       952.694245 &   429.182524 &   1712.206178 &     1400.0 \\
		67 &       988.969597 &   424.452393 &   1693.559656 &     1720.0 \\
		68 &      1170.000553 &   472.174385 &   1883.748679 &     1899.0 \\
	
	\end{tabular}
	
	\begin{figure}
		\centering
		\includegraphics[width=1.3\linewidth]{figs/wykrespredykcji}
		\caption[pred]{Predykcje z przedzialem ufnosci}
		\label{fig:wykrespredykcji}
	\end{figure}
	

\section{Podsumowanie}
	Mimo że model spełnia wszystkie testy i potwierdził początkową hipotezę o zależności ceny od przekątnej, to pokazał on również pewne nieliniowe zachowania wpływające na cenę telewizorów. Zaskakujące okazało się, że jakość nie ma statystyczne znaczącego wpływy na cenę i 80\% można wyjaśnić z wiedzą o stanie i typie telewizora. Same też predykcje z dużym absolutnym odchyłom wskazują na niską użyteczność tego w realnym świecie. Błędy te wynikają po części z małej ilości danych.
\section{Lireratura}

	

\end{document}
